\chapter{Proof of security}

In this chapter, the security of the Stealth Address ZKP Scheme is informally proven.

\section{Assumptions}

\subsection{Groth16}

It is assumed that the pairing-based ZK-SNARK system Groth16 has these properties:
\begin{enumerate}
    \item Completness - honest prover will always convince honest verifier
    \item Soundness - dishonest prover will not convince honest verifier
    \item Zero Knowledge - dishonest verifier will learn nothing more then the truth of the given proposition
\end{enumerate}

If the completness property would not hold, Bob could be locked out of his stealth
addresses, because he could not prove to them that he possessed both secrets.

If the soundness would not hold, adversary could forge a false proof, which
would be verified as true, and thus could gain access to Bob's assets on given
stealth address.

And finally, if the zero knowledge property would not hold, an adversary could
learn Bob's identity, hence it would no longer be a stealth address.

\subsection{Trusted Setup}

The setup phase of Groth16 includes a trusted setup in order to generate the
public parameters. The premise of the trusted setup is that the original parameters
used to create the public ones are thrown away, if not then the party which
produced public parameters can forge false proofs, yet the verifier would
recognize them as valid ones.

In context of this work, if a party creating the circuit didn't delete the
original parameters, then it could prove malicious statements about ownership
of the secret values needed to gain control over any stealth address.


\subsection{Collision resistant hash function}\label{crhf}

First assumption is that, the used instantiation of a hash function (\textit{H})
is a collision resistant one. Meaning it is computationally hard to find two
inputs \textit{a}, \textit{b}, such that

\[ a \neq b \land H(a) = H(b) \]

Otherwise, an adversary could, for example, find a value $x^\prime$,
such that Bob's secret value $x$ and $x^\prime$ are not equal, but their hashes
are. Then the $x^\prime$ could be used to gain control over Bob's stealth address,
if the adversary knew Alice's secret value, or in the case when Alice is a malicious
actor, she could send assets to Bob's stealth address, but with $x^\prime$ and
her secret value, could withdraw them without any public proof that she did it.

As showcased earlier, this vulnerability alone is not enough to gain control
over Bob's stealth address, adversary also must know Alice's secret value to
gain control over a Bob's stealth address, but only the one she interacted with.

Also, if used hash function is not a collision resistant one, adversary still can not
discover the identity of Bob, or any of his other stealth addresses.

\subsection{Elliptic Curve Crypthography - Secp256k1}

Next assumption is that the Secp256k1 elliptic curve parameters used in Ethereum
and Bitcoin\cite{bitcoinSecp256k1Bitcoin} are well-chosen and do not contain any backdoors. Additionally, it is
assumed that advancements in cryptographic research and technology do not
compromise the security of Secp256k1.

If not, an adversary could recover Bob's private key and decrypt all of his
ephemeral keys submitted to the Ephemeral Key Registry contract. The
adversary would know all Bob's stealth addresses and, also have all secret
values used to create codes in those stealth addresses. However, with this
vulnerability alone, the adversary can not control Bob's stealth addresses,
because he/she is missing the Bob's secret value $x$.

\subsection{Knowledge of recipient's secret value}

If an adversary somehow learns Bob's secret value $x$, he/she does not have enough
information to control any of Bob's stealth addresses. On the other hand, if Alice,
or any other sender, would learn Bob's secret value $x$, she could, as in the
\ref{crhf} steal Bob's assets from the stealth address she interacted with and
Bob would not have any public proof that it was her.

\subsection{Knowledge of sender's secret value}

If an adversary somehow learns Alice's secret value $c$, the corresponding
address is still safe, because, the attacker does not know which stealth address
it is, and if somehow learned it, he/she still does not have Bob's secret value.

\section{Proof}

Given these assumptions, the presented Stealth Address ZKP Scheme can be used to
send assets to a recipient, without leaking any information about him/her. And
guarantees to the recipient that only he/she has access to stealth addresses.

