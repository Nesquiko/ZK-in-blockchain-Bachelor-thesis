\chapter{Related work}\label{chapter:related}

With the growth of cryptocurrencies, ZKPs have gained a substantial popularity
not only in academic circles, but also in the startup and venture capital
ones. Many researchers are studying this field and it is not considered 
as a cryptography for nerds anymore. 

On the other hand, there are not many contributions to the stealth address
ecosystem. This field gets some traction, for example, work done by
Fan, Jia and Wang, Zhen and Luo, Yili and Bai, Jian and Li, Yarong and Hao, Yao
\cite{FanJiaWang2019}
proposes new stealth address scheme built on bilinear mappings.

There are two protocols on Ethereum that implement stealth addresses.
First one is called Umbra\cite{umbra}, and it is based on elliptic curve
cryptography.

The second one is called Nocturne\cite{nocturne}. The protocol is a mix of
elliptic curve cryptography and ZKPs with few intermediate smart contracts.
In this protocol you lock your funds in the Nocturne ecosystem, in which
new stealth addresses are created with Elliptic curve stealth addresses scheme,
and you access your funds by submitting ZKPs. However according to their
\href{https://twitter.com/nocturne_xyz/status/1749510390906511693}{Twitter post}
they are discontinuing their protocol.
