% Resume in Slovak
\thispagestyle{empty}

\begin{otherlanguage}{slovak}
\chapter*{Resumé}
\markboth{Resumé}{Resumé}
\addcontentsline{toc}{chapter}{Resumé} 

Táto bakalárska práca skúma aplikáciu dôkazov s nulovým vedomím (ZKPs) v
blockchainoch. Dôkazy s nulovým vedomím sú kryptografická metóda, ktorá
umožňuje overenie dát bez odhalenia samotných dát, čo je ideálne pre
bezpečné a súkromné aplikácie v blockchainoch.

Práca sa zaoberá riešením problému nedostatočného súkromia v
blockchainových transakciách. Blockchainy ako Ethereum ponúkajú
transparentnosť a bezpečnosť, verejná povaha transakcií odhaľuje identity
používateľov a ich finančné aktivity, čo vyvoláva značné obavy o súkromie.
Táto práca predstavuje nové riešenie tohto problému pomocou dôkazov s
nulovým vedomím na vytvorenie schémy tajných adries.

\section{Úvod}

V úvode  je predstavený koncept dôkazov s nulovým vedomím (ZKPs), čo je
kryptografická metóda umožňujúca overenie pravdivosti tvrdenia bez
odhalenia jeho obsahu. Popisuje históriu ZKPs, počnúc ich konceptom v roku
1989, a ich vývoj smerom k neinteraktívnym dôkazom. Tieto dôkazy umožňujú
overenie bez nutnosti priamej interakcie medzi stranami. Kapitola tiež
zdôrazňuje význam tajných adries v blockchainoch, ktoré umožňujú
odosielanie aktív bez odhalenia identity príjemcu. Použitie ZKPs v
kontexte tajných adries zvyšuje súkromie tým, že umožňuje príjemcovi
preukázať vlastníctvo adresy bez prezradenia svojej identity.

Analýza (\ref{chapter:analysis}) opisuje kryptografické princípy za ZKPs a
ich využitie v blockchainoch, najmä v schéme tajných adries. V kapitole
Súvisiaca práca (\ref{chapter:related}) sa skúma existujúci výskum a projekty
týkajúce sa ZKPs a tajných adries. Návrh riešenia (\ref{chapter:solution}) opisuje návrh
konceptu schémy tajných adries využívajúcej ZKPs. V Diskusii o
bezpečnosti (\ref{chapter:security}) sa rozoberajú bezpečnostné aspekty navrhovanej schémy a jej
predpoklady. Kapitola Implementácia (\ref{chapter:implementation}) poskytuje podrobnosti o implementácii
schémy, použitých nástrojoch a technológiách. Kapitola Evaulacia (\ref{chapter:evaluation}) analyzuje
navrhnutú implementáciu z rôznych uhlov, ako napríklad analýza poplatkov, alebo
bezpečnostná analýza.
V kapitole Vyhodnotenie (\ref{chapter:conclusion}) sa
vyhodnocuje implementácia schémy a diskutujú sa jej výsledky. Záver zhŕňa
zistenia práce a navrhuje potenciálne smery pre budúci výskum.

\end{otherlanguage}
