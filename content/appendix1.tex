\setcounter{figure}{0}
\setcounter{listing}{0}

\chapter{Polynomial zero test}
\label{appendix:zero_test}
\pagenumbering{arabic}
\renewcommand*{\thepage}{A-\arabic{page}}

\begin{refsegment}

The polynomial zero test is at the heart of the SNARK construction. Given a
polynomial $f \in \mathbb{F}_p^{\leq d}[X]$, the probability that a random
element $x \in \mathbb{F}_p$ is a root of $f$ is $d/p$.

\[ \Pr[f(x) = 0] \leq \frac{d}{p} \]

For example, $p = 2^{256}$ and $d = 2^{32}$, the probability that a random
element $x \in \mathbb{F}_p$ is a root of $f$ is $2^{224}$. This is a negligible
probability, and we can assume that $f$ is zero polynomial with high
probability when $f(x) = 0$ for a random $x \in \mathbb{F}_p$.

The generalization of the polynomial zero test for multivariate polynomials is
known as \textit{SZDL lemma}.

This can be used to prove that two polynomials are equal. Given two polynomials
$f, g$, we can evaluate $h = f - g$ at a random point $x$. If $h(x) = 0$, then
$f(x) - g(x) = 0 \Longrightarrow f(x) = g(x) \Longrightarrow f = g$ with high probability.

\printbibliography[heading=referencessec,segment=\therefsegment,resetnumbers=true]

\end{refsegment}
