\chapter{Introduction}

Zero Knowledge Proofs (ZKPs) are a powerful cryptography primitive. They allow
for the verification of a statement's truth without disclosing or in any way revealing
the actual content of the statement. This characteristic is crucial for
maintaining trust between parties while also preserving privacy.

The concept of ZKPs was first introduced in a 1989 research paper, "The
Knowledge Complexity of Interactive Proof Systems."\cite{Goldwasser1989}.
This work describes how in traditional proofs, such as demonstrating a graph
is Hamiltonian, more information is typically revealed than just the truth of
the theorem. This paper develops a computational complexity theory focusing
on the knowledge part within a proof. It introduces zero knowledge proofs,
a novel concept where proofs only confirm the correctness of a proposition
without exposing any extra knowledge. The paper focuses on interactive
proofs, where a dialogue between a prover and a verifier occurs. In these
interactive proofs, the prover aims to convince the verifier about the truth
of a private statement, with a very small probability of error. This
interaction is pivotal in ZKPs, as it allows for the verification of a
statement's truth without revealing the actual information or knowledge
behind the statement, maintaining the principle of conveying no
knowledge beyond the proposition's correctness.

This thesis extends the application of ZKPs to the concept of stealth
addresses in blockchain, as outlined in Vitalik Buterin's article "An
Incomplete Guide to Stealth Addresses."\cite{ButerinIncompleteGuide}.
Stealth addresses are critical for privacy on blockchains, allowing assets to
be transferred without revealing the recipient's identity and making
it difficult to link transactions to specific individuals.

Stealth addresses allow a sender (Alice) to transfer assets to a receiver (Bob) without
publicly revealing Bob's identity. To achieve this, Bob
must first provide a public meta stealth address generation data. This data
has different structure based on the underlying stealth address generation
schema. From this data Alice computes a new stealth address that only Bob can
control, and sends the assets to that newly generated address. Bob can then
access these assets from another address, only by providing a ZKP of given
address ownership.

